This thesis investigates the invariant extended Kalman filter (IEKF), a recently introduced method for nonlinear state estimation on matrix Lie groups.
The IEKF is well suited to a particular class of systems, namely those with group-affine process models and invariant measurement models.
In fact, when these conditions are met, the IEKF is a locally asymptotically convergent observer.
However, in practice, process models are often not group-affine, and measurement models are often not invariant.
The effect of removing these assumptions is investigated in this thesis.
In particular, a 3D example is considered, with and without bias estimation.
Estimating bias renders the process model not group affine.
Then, a non-invariant measurement model is considered.
Two different techniques are proposed to incorporate this measurement model into an IEKF, a standard approach using the non-invariant model and a novel approach in which the measurement is preprocessed to force the preprocessed measurement to be invariant.
These practical extensions of the IEKF are tested in simulation to determine the effectiveness of the IEKF for more general state estimation problems.
Lastly, batch estimation in the invariant framework is formulated.
The problem of interest is the simultaneous localization and mapping (SLAM) problem.
A general derivation of the SLAM problem on matrix Lie groups is presented.
Invariant estimation theory is then leveraged. 
An inertial navigation example with bias estimation is then presented, with testing done in simulation.

\cleardoublepage

\newpage
\phantomsection
\hbox{ }
\twoinmar
\selectlanguage{french}
\centerline{\large\bf R\'esum\'e}
\vspace{0.7in}
\onehalfspacing

Cette thèse étudie le filter de Kalman invariant (IEKF), une méthode récemment introduite pour l'estimation d'état non linéaire sur des groupes de Lie matriciels. 
L'IEKF est bien adapté à une classe particulière de systèmes, plus précisément ceux dotés de fonctions affines et de fonctions d'observation invariantes. 
En fait, lorsque ces conditions sont satisfaites, l'IEKF est un observateur localement asymptotiquement convergent. 
Cependant, en situation pratique, ces conditions ne sont souvent pas satisfaites. 
Des scénarios ou ces conditions ne sont pas satisfaites sont étudié ici. 
En particulier, un exemple 3D est considéré, avec et sans estimation de biais dans le gyroscope. 
L'estimation du biais rend la fonction  non-affinée. 
Ensuite, une fonction d'observation non-invariante est considéré. 
Deux techniques différentes sont proposées pour incorporer cette fonction d'observation dans un IEKF, une approche standard utilisant la fonction non-invariante et une nouvelle approche dans laquelle la mesure est prétraitée pour la forcer être invariante. 
Ces extensions pratiques du IEKF sont testées en simulation pour déterminer son efficacités pour des problèmes d'estimation d'état plus généraux. 
Enfin, une technique d'estimation par lot dans le cadre invariant est formulée. 
Un intérêt particulier est porté au problème de la localisation et de la cartographie simultanées (SLAM). 
Une dérivation générale du problème SLAM sur les groupes de Lie matriciels est présentée. 
La théorie de l'estimation invariante est ensuite mise à profit. 
Un exemple de navigation inertielle avec estimation du biais est ensuite présenté, avec des tests effectués en simulation.

\selectlanguage{english}