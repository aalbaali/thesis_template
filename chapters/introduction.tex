High-level introduction to the field of research.

%-------------------------------------------------------------------------------
\section{Thesis Objective}

%-------------------------------------------------------------------------------
\section{Thesis Overview}

This thesis is structured as follows.

\Cref{chap:preliminaries} summarizes mathematical concepts and notation that are used throughout this thesis. 

\verb+\Cref{chap:iekf}>+ outlines the IEKF. The relevant theorems and proofs are presented in continuous and discrete-time. The left-invariant extended Kalman filter and right-invariant extended Kalman filter are then detailed.

In \verb+\cref{chap:SE3}>+, several examples of the IEKF are presented to illustrate how to practically implement an IEKF and to compare its performance to that of a standard multiplicative extended Kalman filter (MEKF). 

In \verb+\cref{chap:batch}>+, a solution to the SLAM problem in the invariant framework is presented. Simulation results are shown comparing the novel formulation to more traditional batch-based solutions to the SLAM problem.

This thesis is concluded in \cref{chap:conclusion}, where a summary of the findings are presented, along with recommended future work.


% Remove this section when writing the thesis
\section{TEMPLATE: Using \texttt{cleveref} and \texttt{acronym}}
A system of equations is given by
\begin{align}
  \label{eq:Ax=b}
  \mbf{A} \mbf{x} 
    &= 
    \mbf{b}.
\end{align}
Use \verb+\cref+ to reference a label (equation, figure, table, chapter, \etc) in the middle of a sentence, and use \verb+\Cref+ to reference a label at the beginning of the sentence.

For example, \cref{eq:Ax=b} is an important equation.
\Cref{eq:Ax=b} is used extensively in estimation theory (note how `Equation' is automatically added when using \verb+\Cref+).

The \texttt{acronym} package handles acronyms well using \verb+\Ac+ (at the beginning of the sentence) and \verb+\ac+ (otherwise).
The package will take care of expanding the acronym when it's first used, and then uses the abbreviation/short-name afterwards.

For example, \ac{SLAM} is an interesting field of research.
Probability is extensively used in \ac{SLAM}.
The upper-case/lower-case rules of the long names (defined in \texttt{abbreviations.tex}) should be consistent for all acronyms.